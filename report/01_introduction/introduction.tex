\section{Introduction}
This assignment involves employing genetic programming to produce a classifier for postoperative patient diagnosis.

More specifically, the toolset provided by ECJ \cite{luke2006ecj} - a java-based evolutionary computation research system - was used to construct a genetic programming approach to evolve a population of decision trees for classifying postoperative patient diagnosis for the patients provided in the Post-Operative Patient Data Set, which can be obtained from the UCI Machine Learning repository \cite{Dua:2019}.

The aforementioned data set consist of 90 records, each with 9 attributes which can be described as follows:

\begin{itemize}
    \item L-CORE (patient's internal temperature in C):
              high ($>$ 37), mid ($\geq$ 36 and $\leq$ 37), low ($<$ 36)
    \item L-SURF (patient's surface temperature in C):
              high ($>$ 36.5), mid ($\geq$ 36.5 and $\leq$ 35), low ($<$ 35)
    \item L-O2 (oxygen saturation in %):
              excellent ($\geq$ 98), good ($\geq$ 90 and $<$ 98),
              fair ($\geq$ 80 and $<$ 90), poor ($<$ 80)
    \item L-BP (last measurement of blood pressure):
              high ($>$ 130/90), mid ($\leq$ 130/90 and $\geq$ 90/70), low ($<$ 90/70)
    \item SURF-STBL (stability of patient's surface temperature):
              stable, moderate, unstable
    \item CORE-STBL (stability of patient's core temperature)
              stable, moderate, unstable
    \item BP-STBL (stability of patient's blood pressure)
              stable, moderate, unstable
    \item COMFORT (patient's perceived comfort at discharge, measured as
              an integer between 0 and 20)
    \item ADM-DECS (discharge decision):
              I (patient sent to Intensive Care Unit),
              S (patient prepared to go home),
              A (patient sent to general hospital floor)
\end{itemize}

The classification task of this data set is to determine where patients in a postoperative recovery area should be sent to next (the discharge decision). For this assignment, the \emph{COMFORT} attribute is not considered for the classification problem as it contains missing values. So, the classifier, in the form of a decision tree, which is produced will use 7 patient attributes to classify patient's into one of three discharge decisions.

All of the code that was used to employ the genetic algorithm can be found on github at \url{https://github.com/marcus-bornman/cos_710_assignment_2}. The repository also contains all information necessary to reproduce the results that are discussed within this assignment.
