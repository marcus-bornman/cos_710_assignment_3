\section{Introduction}
This assignment involves employing genetic programming and incorporating modularization using automatically defined functions (ADFs) to produce a classifier for postoperative patient diagnosis. In addition, the results of this approach are compared to those produced by a previous genetic programming approach that did not incorporate modularization - see \url{https://github.com/marcus-bornman/cos_710_assignment_2}.

The toolset provided by ECJ \cite{luke2006ecj} - a java-based evolutionary computation research system - was used to construct the genetic programming approach to evolve a population of arrays of trees. The first tree in each array represents the main decision-tree for classifying postoperative patient diagnosis for the patients in the Post-Operative Patient Data Set (available from the UCI Machine Learning repository \cite{Dua:2019}). The rest of the trees in each array represent ADFs.

The aforementioned data set consist of 90 records. The data set was randomly divided into a training set of 70 records and an evaluation set of 20 records. Each record consists of 9 attributes, which can be described as follows:

\begin{itemize}
    \item L-CORE (patient's internal temperature in C):
              high ($>$ 37), mid ($\geq$ 36 and $\leq$ 37), low ($<$ 36)
    \item L-SURF (patient's surface temperature in C):
              high ($>$ 36.5), mid ($\geq$ 36.5 and $\leq$ 35), low ($<$ 35)
    \item L-O2 (oxygen saturation in %):
              excellent ($\geq$ 98), good ($\geq$ 90 and $<$ 98),
              fair ($\geq$ 80 and $<$ 90), poor ($<$ 80)
    \item L-BP (last measurement of blood pressure):
              high ($>$ 130/90), mid ($\leq$ 130/90 and $\geq$ 90/70), low ($<$ 90/70)
    \item SURF-STBL (stability of patient's surface temperature):
              stable, moderate, unstable
    \item CORE-STBL (stability of patient's core temperature)
              stable, moderate, unstable
    \item BP-STBL (stability of patient's blood pressure)
              stable, moderate, unstable
    \item COMFORT (patient's perceived comfort at discharge, measured as
              an integer between 0 and 20)
    \item ADM-DECS (discharge decision):
              I (patient sent to Intensive Care Unit),
              S (patient prepared to go home),
              A (patient sent to general hospital floor)
\end{itemize}

The classification task of this data set is to determine where patients in a postoperative recovery area should be sent to next (the discharge decision). For this assignment, the \emph{COMFORT} attribute is not considered for the classification problem as it contains missing values. So, the classifier will use 7 patient attributes to classify patient's into one of three discharge decisions.

All of the code that was used to employ the genetic algorithm and the information necessary to reproduce the results can be found on github at \url{https://github.com/marcus-bornman/cos_710_assignment_3}.
